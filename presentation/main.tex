
\documentclass[aspectratio=169]{bredelebeamer}

\usepackage{tikz}
\usepackage[backend=bibtex]{biblatex}
\setbeamertemplate{bibliography item}[text]
\usepackage[font={scriptsize}]{caption}
\usepackage{booktabs}
\bibliography{bibliografia.bib}




%%%%%%%%%%%%%%%%%%%%%%%%%%%%%%%%%%%%%%%%%%%%%%%%

\title{HACC\\}

\subtitle{Hardware Accelerated Cosmology Code}


\author{
\textbf{PPTM - Q2 2015} \\
Professor: Jesus Labarta \\
Author: Cristóbal Ortega}
%\date{\today}
\date{June 23, 2015}

%%%%%%%%%%%%%%%%%%%%%%%%%%%%%%%%%%%%%%%%%%%%%%%%%%%%%%%%%%%%%%%%%%%%%
\begin{document}

\begin{frame}[plain]
  \titlepage
\end{frame}


\begin{frame}{Outline}
  \tableofcontents
\end{frame}



\section{HACC}
\subsection{HACC }
\begin{frame}{Overview\footfullcite{HACC_paper}}
\begin{itemize}

\item Getting the N-body problem to petaflop systems
\begin{itemize}
	\item "The N-body problem is the problem of predicting the individual motions of a group of celestial objects interacting with each other gravitationally"
\end{itemize}

\end{itemize}

\begin{columns}
\hspace*{0.5cm}
\column{0.6\textwidth}

\begin{itemize}

\item HACC uses a hybrid parallel algorithmic structure:
\begin{itemize}
	\item A grid-based for long/medium range (common in all these codes)
	\item An architecture-tunable particle-based short/close-range solver
\end{itemize}

\end{itemize}
\column{0.4\textwidth}
\begin{figure}[h!]
  \centering
\includegraphics[width=0.65\textwidth, keepaspectratio=1]{"./img/introduction/hacc_force_evaluation"}
\caption*{Green: medium/short range\\Magenta: close range}
\end{figure}

\end{columns}

\begin{itemize}


\item Coded to run in large systems: \\
 Tested on 1,572,864 cores with 90\% parallel efficiency
\end{itemize}
\end{frame}


\begin{frame}{Input}

Important parameters to consider:

\begin{itemize}
\item N particles alive
\item Steps
\item Geometry:
	\begin{itemize}
		\item Virtual topology of the network
		\item D1xD2xD3
		\item Each dimension must of the decomposition must be a deviser of the N value.
		\item A Cartesian topology will be created based on it
	\end{itemize}
\end{itemize}        

\end{frame}



\begin{frame}{Topology}

In the presentation 2 types of topologies will be analyzed:\\
\vspace{0.5cm}
\textbf{Distributed}
\begin{itemize}
	\item Per a given number of MPI processors it will be decomposed in powers of 2 across all the dimensions
	\begin{itemize}
		\item For 2 ranks $\rightarrow$ 2x1x1
		\item For 4 ranks $\rightarrow$ 2x2x1
		\item For 8 ranks $\rightarrow$ 2x2x2
		\item For 64 ranks $\rightarrow$ 4x4x4
		\item For 128 ranks $\rightarrow$ 8x4x4
	\end{itemize}
\end{itemize}
\vspace{0.5cm}
\textbf{Non Distributed}
\begin{itemize}
	\item Per a given number of MPI processors all of them will correspond to the first dimension:
	\begin{itemize}
		\item For 2 ranks $\rightarrow$ 2x1x1
		\item For 4 ranks $\rightarrow$ 4x1x1
		\item For 8 ranks $\rightarrow$ 8x1x1
		\item For 64 ranks $\rightarrow$ 64x1x1
		\item For 128 ranks $\rightarrow$ 128x1x1
	\end{itemize}
\end{itemize}


\end{frame}


\section{Analysis}

\subsection{Basic analysis}
\begin{frame}{Strong Scalability}
\begin{itemize}
	\item Shown only for testing behaviour
	\item Will focus on the weak scalability
	\item Execution time for 1 process: 5 hours
\end{itemize}

\begin{columns}

\column{0.5\textwidth}
\begin{figure}[h!]
\centering
\includegraphics[width=6cm,height=5cm, keepaspectratio=1]{"./img/introduction/strong_scalability_speedup"}
\end{figure}

\column{0.5\textwidth}
\begin{figure}[h!]
\centering
\includegraphics[width=6cm,height=5cm, keepaspectratio=1]{"./img/introduction/strong_scalability_efficiency"}
\end{figure}


\end{columns}

\end{frame}

\begin{frame}{Strong Scalability (II)}
\begin{itemize}
	\item Similar behaviour
	\item They run with 1.073.741.824 particles
\end{itemize}

\begin{columns}

\column{0.5\textwidth}
\begin{figure}[h!]
\centering
\includegraphics[width=6cm,height=5cm, keepaspectratio=1]{"./img/introduction/strong_scalability_efficiency_authos"}
\end{figure}

\column{0.5\textwidth}
\begin{figure}[h!]
\centering
\includegraphics[width=6cm,height=5cm, keepaspectratio=1]{"./img/introduction/strong_scalability_efficiency"}
\end{figure}


\end{columns}

\end{frame}


\begin{frame}{Weak Scalability}


\begin{itemize}
\item A little imbalance by input\\
Problem size has to be multiple of number of processors\\
\item For a topology 4x4x2, the problem size has to be multiple of 4$\times$4$\times$2 \linebreak 

\item $\text{Performance} = \frac{ \text{Execution Time[8 MPI Rank}]}{\text{Execution Time[N MPI Ranks]} } \times \frac{ \text{Problem Size[N MPI Rank}]}{\text{Problem Size[8 MPI Ranks]} }$ 
\end{itemize}


\begin{columns}

\column{0.5\textwidth}
\begin{figure}[h!]
\centering
\includegraphics[width=\textwidth, keepaspectratio=1]{"./img/basic_analysis/performance_distributed"}
\caption*{Performance for Distributed topologies}
\end{figure}

\column{0.5\textwidth}
\begin{figure}[h!]
\centering
\includegraphics[width=\textwidth, keepaspectratio=1]{"./img/basic_analysis/performance_nonDistributed"}
\caption*{Performance for non Distributed topologies}
\end{figure}


\end{columns}

\end{frame}


\begin{frame}{Weak Scalability (II)}

\begin{columns}

\column{0.5\textwidth}
\begin{figure}[h!]
\centering
\includegraphics[width=\textwidth, keepaspectratio=1]{"./img/basic_analysis/modelFactors_distributed"}
\end{figure}

\begin{figure}[h!]
\centering
\includegraphics[width=\textwidth, keepaspectratio=1]{"./img/basic_analysis/overallEfficiency_distributed"}
\end{figure}

\column{0.5\textwidth}
\begin{figure}[h!]
\centering
\includegraphics[width=\textwidth, keepaspectratio=1]{"./img/basic_analysis/modelFactors_nonDistributed"}
\end{figure}

\begin{figure}[h!]
\centering
\includegraphics[width=\textwidth, keepaspectratio=1]{"./img/basic_analysis/overallEfficiency_nonDistributed"}
\end{figure}


\end{columns}

\end{frame}




\subsection{General Structure}
\begin{frame}{Overview}

\begin{columns}
\column{0.25\textwidth}


	\begin{itemize}
	\item 3 Steps
	\item 4x4x2 config
	\item MPI Calls
	\item Useful duration
	\end{itemize} 
	
	
\column{0.75\textwidth}

\begin{figure}[h!]
  \centering
    \includegraphics[width=\textwidth,height=5cm,keepaspectratio]{"./img/general/useful_mpi_unbalanced"}
\end{figure}


\end{columns}


\begin{alertblock}{Imbalance}
\begin{itemize}
\item Some process waiting in MPI to others to finalize
\end{itemize}
\end{alertblock}

\end{frame}

\begin{frame}{For different configurations}
\begin{columns}[c]
\column{0.1\textwidth}


2x2x2\\
\vspace{1.25cm}
4x2x2\\
\vspace{1cm}
4x4x2\\
\vspace{1cm}
4x4x4\\
\vspace{1cm}
8x4x4




\column{0.4\textwidth}

\includegraphics[height=1.5cm, width=4.5cm]{"./img/general/2x2x2"}


\includegraphics[height=1.5cm, width=4.5cm]{"./img/general/4x2x2"}


\includegraphics[height=1.5cm, width=4.5cm]{"./img/general/4x4x2"}


\includegraphics[height=1.5cm, width=4.5cm]{"./img/general/4x4x4"}


\includegraphics[height=1.5cm, width=4.5cm]{"./img/general/8x4x4"}


\column{0.1\textwidth}


8x1x1 \\
\vspace{1.25cm}
16x1x1\\
\vspace{1cm}
32x1x1\\
\vspace{1cm}
64x1x1\\
\vspace{1cm}
128x1x1\\


\column{0.4\textwidth}
\includegraphics[height=1.5cm, width=4.5cm]{"./img/general/8x1x1"}


\includegraphics[height=1.5cm, width=4.5cm]{"./img/general/16x1x1"}


\includegraphics[height=1.5cm, width=4.5cm]{"./img/general/32x1x1"}


\includegraphics[height=1.5cm, width=4.5cm]{"./img/general/64x1x1"}


\includegraphics[height=1.5cm, width=4.5cm]{"./img/general/128x1x1"}



\end{columns}

\end{frame}


\subsection{Tracking}
\begin{frame}{Notes}
\begin{itemize}
	\item From 8 processes up to 128
	\item Showing distributed and non-distributed topologies
	\begin{itemize}
		\item Differences between number of instructions
	\end{itemize}
	\item \textbf{64 processes} case has a smaller input data 
\end{itemize}
\end{frame}


\begin{frame}{Evolution}
\begin{columns}

\column{0.5\textwidth}
	\begin{figure}[h!]
	\centering
	\includegraphics<1>[width=\textwidth, keepaspectratio]{"./img/tracking/8_d"}
	\includegraphics<2>[width=\textwidth, keepaspectratio]{"./img/tracking/16_d"}
	\includegraphics<3>[width=\textwidth, keepaspectratio]{"./img/tracking/32_d"}
	\includegraphics<4>[width=\textwidth, keepaspectratio]{"./img/tracking/64_d"}
	\includegraphics<5>[width=\textwidth, keepaspectratio]{"./img/tracking/128_d"}

	\caption*{
		\only<1>{\#Instructions vs IPC, 8 ranks, 2x2x2}
		\only<2>{\#Instructions vs IPC, 16 ranks, 4x2x2}
		\only<3>{\#Instructions vs IPC, 32 ranks, 4x4x2}
		\only<4>{\#Instructions vs IPC, 64 ranks, 4x4x4}
		\only<5>{\#Instructions vs IPC, 128 ranks, 8x4x4}	
	}
	\end{figure}

\column{0.5\textwidth}
	\begin{figure}[h!]
	\centering
	\includegraphics<1>[width=\textwidth, keepaspectratio]{"./img/tracking/8_nd"}
	\includegraphics<2>[width=\textwidth, keepaspectratio]{"./img/tracking/16_nd"}
	\includegraphics<3>[width=\textwidth, keepaspectratio]{"./img/tracking/32_nd"}
	\includegraphics<4>[width=\textwidth, keepaspectratio]{"./img/tracking/64_nd"}
	\includegraphics<5>[width=\textwidth, keepaspectratio]{"./img/tracking/128_nd"}	
	
	\caption*{
		\only<1>{\#Instructions vs IPC, 8 ranks, 8x1x1}
		\only<2>{\#Instructions vs IPC, 16 ranks, 16x1x1}
		\only<3>{\#Instructions vs IPC, 32 ranks, 32x1x1}
		\only<4>{\#Instructions vs IPC, 64 ranks, 64x1x1}
		\only<5>{\#Instructions vs IPC, 128 ranks, 128x1x1}	
	}
	\end{figure}

\end{columns}

\end{frame}



\begin{frame}{Metrics}
\begin{itemize}
\item Instructions:
	\begin{itemize}
		\item Drop in 64 ranks by the given input, no application problem!
		\item Very similar between ranks (most imbalance around 5-10\%)
	\end{itemize}
\item IPC:
	\begin{itemize}
		\item Region 1. Computation: IPC keeps equal
		\item Region 2 \& 3. More imbalance if there is no distribution in the space.
	\end{itemize}
\end{itemize}
\centering
\includegraphics[width=8cm, height=4cm, keepaspectratio]{"./img/tracking/instr_all_regions"}


\centering
\includegraphics[width=8cm, height=4cm, keepaspectratio]{"./img/tracking/IPC_all_regions"}



\end{frame}


\subsection{One Step Structure}
\begin{frame}{Overview}

\begin{columns}[c]
\column{0.25\textwidth}


	\begin{itemize}
	\item 4x4x2 config
	\item 32 MPI Ranks
	\end{itemize}
	\begin{block}{Phases}
		\begin{enumerate}						
		\item \textcolor{yellow}{ Init}
		\item \textcolor{Green}{Computation}
		\item \textcolor{red}{End}
		\end{enumerate}	
	\end{block}
 


\column{0.75\textwidth}

\begin{figure}[h!]
  \centering
    \includegraphics[width=\textwidth,height=5cm,keepaspectratio]{"./img/one_step/phases"}
\end{figure}

\end{columns}
\end{frame}


\begin{frame}{Clustering}
\includegraphics[width=\textwidth, keepaspectratio=1]{"./img/one_step/clustering"}


\end{frame}

\begin{frame}{Imbalance}
\begin{itemize}

\item Cluster of computation shows vertical line\\
    $\rightarrow$ Difference in number of instructions\\
\item Losing performance:
\end{itemize}


\begin{columns}[c]
\column{0.33\textwidth}
% Please add the following required packages to your document preamble:
% \usepackage{booktabs}
% \usepackage[table,xcdraw]{xcolor}
% If you use beamer only pass "xcolor=table" option, i.e. \documentclass[xcolor=table]{beamer}
\begin{table}[h]
\centering
\begin{tabular}{ll}
\toprule
        & Outside MPI \\ \midrule
\rowcolor[HTML]{EFEFEF} 
Avg     & 94,85\%     \\
Max     & 99,95\%     \\
\rowcolor[HTML]{EFEFEF} 
Min     & 89,07\%     \\
Avg/Max & 0,95\%      \\ \bottomrule
\end{tabular}
\caption*{4x4x2}
\label{mpi_4x4x2}
\end{table}


\column{0.33\textwidth}

\begin{table}[h]
\centering
\begin{tabular}{ll}
\toprule
        & Outside MPI \\ \midrule
\rowcolor[HTML]{EFEFEF} 
Avg     & 92,14\%     \\
Max     & 99,93\%     \\
\rowcolor[HTML]{EFEFEF} 
Min     & 86,70\%     \\
Avg/Max & 0,95\%      \\ \bottomrule
\end{tabular}
\caption*{4x4x4}
\label{mpi_4x4x4}
\end{table}


\column{0.33\textwidth}

\begin{table}[h]
\centering
\begin{tabular}{ll}
\toprule
        & Outside MPI \\ \midrule
\rowcolor[HTML]{EFEFEF} 
Avg     & 96,46\%     \\
Max     & 99,86\%     \\
\rowcolor[HTML]{EFEFEF} 
Min     & 93,80\%     \\
Avg/Max & 0,97\%      \\ \bottomrule
\end{tabular}
\caption*{128x1x1}
\label{mpi_128x1x1}
\end{table}

\end{columns}


\end{frame}







\subsection{Source of imbalance}
\begin{frame}{IPC}

\begin{itemize}
\item Computation phase with same IPC
\item Blue phases are the important ones, computation phase
\end{itemize}

\begin{figure}[h!]
  \centering
  \includegraphics[width=\textwidth,height=4cm,keepaspectratio]{"./img/one_step/IPC_histograma_4x4x2"}
  \caption*{IPC Histogram}
\end{figure}

\begin{itemize}
\item Min. IPC: 1.50
\item Max. IPC: 1.57
\end{itemize}

\end{frame}

\begin{frame}{Instructions}

\begin{itemize}
\item IPC = $\frac{\#Instructions}{\#Cycles}$
\item Blue phases are the important ones, computation phase
\end{itemize}

\begin{figure}[h!]
  \centering
  \includegraphics[width=\textwidth,height=4cm,keepaspectratio]{"./img/one_step/instructions_histograma_4x4x2"}
  \caption*{Instructions Histogram}
\end{figure}

\begin{itemize}
\item Min. Instructions: $7.5\times10^{12}$
\item Max. Instructions: $8.5\times10^{12}$
\end{itemize}

\end{frame}


\begin{frame}{Preemption}

\begin{itemize}
\item Extra instructions come from preemption?
\item Blue phases are the important ones, computation phase
\end{itemize}

\begin{figure}[h!]
  \centering
  \includegraphics[width=\textwidth,height=4cm,keepaspectratio]{"./img/one_step/CYCLES-US_histograma_4x4x2"}
  \caption*{Cycles per Us Histogram}
\end{figure}

\begin{itemize}
\item No preemption in the computing phases
\end{itemize}

\end{frame}




\begin{frame}{Network}

\begin{itemize}
\item  Is the Network affecting the application?\\
\end{itemize}

\begin{columns}

\column{0.25\textwidth}
\begin{block}{Ideal network}
Bandwidth: $\infty$ MB$/$s \\
Latency: 0s \\
\end{block}
\end{columns}

\vspace{0.2cm}
\begin{columns}

\column{0.5\textwidth}
\begin{figure}[h!]
  \centering
\includegraphics[width=6cm, keepaspectratio=1]{"./img/one_step/dimemas/dimemas_4x2x2"}
\caption*{Dimemas: ideal network execution}
\end{figure}



\column{0.5\textwidth}
\begin{figure}[h!]
  \centering
\includegraphics[width=6cm, keepaspectratio=1]{"./img/one_step/dimemas/against_dimemas_4x2x2"}
\caption*{Real execution}
\end{figure}

\end{columns}
\end{frame}


\subsection{Special cases}
\begin{frame}{Why less instructions}

\begin{itemize}
\item 64 \& 128 processes
\begin{itemize}
	\item Same input
	\item Different virtual topology
	\item Less instructions
\end{itemize}
\item Focus on 128 processes case

\end{itemize}

\end{frame}



\begin{frame}{Output}

\begin{columns}

\column{0.5\textwidth}

\begin{itemize}
\item 8x4x4
\item[] Initial Particles 16777216 
\begin{itemize}
	\item 131072 particles per process
\end{itemize}
\item[] STEP 0: particles = 16777207
\item[] STEP 1: particles = 16777215
\item[] STEP 2: particles = 16777213
\end{itemize}

\column{0.5\textwidth}

\begin{itemize}
\item 128x1x1
\item[] Initial Particles 17307773 

\begin{itemize}
	\item 135216 particles per process
\end{itemize}
\item[] STEP 0: particles = 17332773
\item[] STEP 1: particles = 17363514
\item[] STEP 2: particles = 17404656
\end{itemize}


\end{columns}

\vspace{1cm}

\begin{itemize}
\item Size of the generated traces:
\begin{itemize}
	\item 8x4x4: 200 MB
	\item 128x1x1: 1 GB
\end{itemize}
\end{itemize}


\end{frame}


\section{Conclusions}

\subsection{}

\begin{frame}{Conclusions}
\begin{itemize}
\item HACC is a bit imbalance (up to ~10\%)
\item Seems to be cause of how the algorithm works
	\begin{itemize}
		\item Input points are distributed among the different process\\
		\item There are no hardware problems\\
	\end{itemize}
\end{itemize}	

\hspace{1cm}
%\begin{itemize}
%\item Possible solutions:
%\begin{itemize}
%	\item Try with a bigger problem $\rightarrow$ ¿better division of the problem?
%	\item Increase number of MPI process working $\rightarrow$ ¿better distribution of the problem?
%	\item Execute with more threads $\rightarrow$ dynamic scheduling could improve perfomance	
%	\item Dynamic Load Balancing $\rightarrow$ assign more resources to those processes with more instructions
%\end{itemize}
%\end{itemize}

\begin{itemize}
\item Possible solutions:
\begin{itemize}
	\item Try with a bigger problem 
	\begin{itemize}
		\item[] $\rightarrow$ ¿better division of the problem?
	\end{itemize}
	
	
	\item Increase number of MPI process working
	\begin{itemize}
		\item[] $\rightarrow$ ¿better distribution of the problem?
	\end{itemize} 
	
	\item Execute with more threads
	\begin{itemize}
		\item[]  $\rightarrow$ dynamic scheduling could improve perfomance
	\end{itemize} 
	
	\item Dynamic Load Balancing 
		\begin{itemize}
		\item[]  $\rightarrow$ assign more resources to those processes with more instructions
	\end{itemize} 
\end{itemize}
\end{itemize}

\end{frame}

\begin{frame}{Problems}
\begin{itemize}
\item Did not manage to get threads working $\rightarrow$ It could help to balance number of instructions
\begin{itemize}
	\item It seemed there was a race condition
\end{itemize}
\vspace{0.5cm}
\item Spawned a lot of threads
	\begin{itemize}
		\item There is a nested loop with OpenMP
		\item With only OMP\_NUM\_THREADS=X, it will create more
		\item OMP\_THREAD\_LIMIT=1 is needed
	\end{itemize}
\end{itemize}


\end{frame}





\subsection{}
\begin{frame}{References}

   \begin{block}{HACC}
      \printbibliography[heading=none]
   \end{block}
   
\begin{block}{Tools}
\begin{itemize}
\item Perfomance tools documentation: \\http://www.bsc.es/computer-sciences/performance-tools/documentation

\item Constan documentation

\end{itemize}
   \end{block}
\end{frame}

\subsection{}
\begin{frame}{Questions}
\centering

\boitegrise{
\centering \Large Thank you!\\ Feel free to ask\\
\vspace{0.5cm}
\normalsize Cristobal Ortega \small cristobal.ortega@est.fib.upc.edu
}

\end{frame}



\end{document}
